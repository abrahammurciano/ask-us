\subsection{Constraints}

\subsubsection{Adding Constraints}

We are tasked with creating and applying any constraints we deem appropriate to our database. We have already applied all appropriate unique constraints when creating the table, now we will go over all our tables and add check constraints where necessary.

For the users table, we must check that the email address provided is well formed. Meaning that it contains some text, followed by an @ symbol, followed by a domain. We also must ensure that the points are always positive. Here is the alter table command for the users table.

\VerbatimInput[label=\fbox{\color{Black}/sql/constraints/users.sql}]{../sql/constraints/users.sql}

When looking at the questions, answers, and comments table, the only field which needed to be checked was once again the points field. Below are the alter table commands for all three.

\VerbatimInput[label=\fbox{\color{Black}/sql/constraints/questions.sql}]{../sql/constraints/questions.sql}
\VerbatimInput[label=\fbox{\color{Black}/sql/constraints/answers.sql}]{../sql/constraints/answers.sql}
\VerbatimInput[label=\fbox{\color{Black}/sql/constraints/comments.sql}]{../sql/constraints/comments.sql}

\subsubsection{Testing the Constraints}

To test that all our constraints are working correctly, we ran the following commands which would, if successful, violate some of our constraints. First we tried to add a user with an invalid email address.

\VerbatimInput[label=\fbox{\color{Black}/sql/constraints/tests/users.email.sql}]{../sql/constraints/tests/users.email.sql}

This command was unsuccessful and gave the following error message. The reason for this is because the email address provided does not have a valid domain, since there is no dot (.) character.

\begin{verbatim}
	ORA-02290: check constraint (AMURCIAN.CHECK_USERS_EMAIL) violated
\end{verbatim}

Subsequently, we ran an update query to ensure that the points field was not able to become negative.

\VerbatimInput[label=\fbox{\color{Black}/sql/constraints/tests/questions.points.sql}]{../sql/constraints/tests/questions.points.sql}

Similarly, we obtained an error message as follows, since we do not allow negative values in the points field.

\begin{verbatim}
	ORA-02290: check constraint (AMURCIAN.CHECK_QUESTIONS_POINTS)
	violated
\end{verbatim}

We also checked that unique constraints could not be violated by attempting to update a user's username to an already existing one.

\VerbatimInput[label=\fbox{\color{Black}/sql/constraints/tests/users.username.sql}]{../sql/constraints/tests/users.username.sql}

This time, we received a slightly different error message. The reason for the error was because there was already an existing user with the username `werlwend'.

\begin{verbatim}
	ORA-00001: unique constraint (AMURCIAN.SYS_C00574442) violated
\end{verbatim}

Finally, we tested a foreign key constraint by deleting a post from the post table, which had both a child comment and its question record referencing it.

\VerbatimInput[label=\fbox{\color{Black}/sql/constraints/tests/posts.id.sql}]{../sql/constraints/tests/posts.id.sql}

Once again, as expected, we got an error message, because if the record would have been deleted, there would be a question whose post id would not be in the post table, and there would be a comment whose parent post would not exist.

\begin{verbatim}
	ORA-02292: integrity constraint (AMURCIAN.SYS_C00574679) violated
	 - child record found
\end{verbatim}
