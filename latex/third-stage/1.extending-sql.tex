\subsection{Extending SQL}

This stage is about using PL/SQL to extend the SQL we have used so far. We have two tasks for this section. Firstly, we must write a script that uses PL/SQL and demonstrates a cursor loop, and before and after triggers.

First of all, our script uses a cursor loop to display all the topics that each user follows.

\VerbatimInput[label=\fbox{\color{Black}/sql/plsql/loop.sql}]{../sql/plsql/loop.sql}

This script gives us an output like this. (Only the first ten lines are shown here.)

\VerbatimInput[label=\fbox{\color{Black}/sql/plsql/loop\_output.txt}]{../sql/plsql/loop_output.txt}

Then our script uses before insert and update triggers to store the given password as the base 64 encoded SHA256 hash, rather than as the given plain text.

\VerbatimInput[label=\fbox{\color{Black}/sql/plsql/triggers/hash\_password.sql}]{../sql/plsql/triggers/hash_password.sql}

Following this, we tested the trigger with an insert and an update statement (shown below), and they both resulted in storing the correct hashed password.

\VerbatimInput[label=\fbox{\color{Black}/sql/plsql/triggers/test\_hash\_password.sql}]{../sql/plsql/triggers/test_hash_password.sql}

For our `after' trigger, we found that whenever someone votes on a post, as well as inserting a record into the votes table, we also need to increment the points of the post that was voted on, as well as the points of the author of the post. We therefore created a trigger on the votes table that would do this automatically.

\VerbatimInput[label=\fbox{\color{Black}/sql/plsql/triggers/points\_distributer.sql}]{../sql/plsql/triggers/points_distributer.sql}

We then tested it with the insert statement shown below, and indeed, the post and the user both received an additional point.

\VerbatimInput[label=\fbox{\color{Black}/sql/plsql/triggers/test\_points\_distributer.sql}]{../sql/plsql/triggers/test_points_distributer.sql}
