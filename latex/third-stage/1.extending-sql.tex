\subsection{Extending SQL}

This stage is about using PL/SQL to extend the SQL we have used so far. We have two tasks for this section. Firstly, we must write a script that uses PL/SQL and demonstrates a cursor loop, and before and after triggers.

First of all, our script uses a cursor loop to display all the topics that each user follows.

\VerbatimInput[label=\fbox{\color{Black}/sql/plsql/loop.sql}]{../sql/plsql/loop.sql}

This script gives us an output like this. (Only the first ten lines are shown here.)

\VerbatimInput[label=\fbox{\color{Black}/sql/plsql/loop\_output.txt}]{../sql/plsql/loop_output.txt}

Then our script uses before insert and update triggers to store the given password as the base 64 encoded SHA256 hash, rather than as the given plain text.

\VerbatimInput[label=\fbox{\color{Black}/sql/plsql/triggers/hash\_password.sql}]{../sql/plsql/triggers/hash_password.sql}

Following this, we tested the trigger with an insert and an update statement (shown below), and they both resulted in storing the correct hashed password.

\VerbatimInput[label=\fbox{\color{Black}/sql/plsql/triggers/test\_hash\_password.sql}]{../sql/plsql/triggers/test_hash_password.sql}
