% for text un key attributes
\newcommand{\key}[1]{\underline{#1}}
\newcommand{\fkey}[1]{\textit{#1}}

\section{Logical Schema}

We are to convert the ERD in the previous section into a logical schema. First we will convert the entities into tables, and then we will add tables for some of the relations where required.

See the entity tables below. An underlined value indicates a primary key. An \emph{Italicized} value indicates a foreign key.

\begin{itemize}
	\item[] \textsc{User} (\key{ID}, Name, Email, Password, Points)
	\item[] \textsc{Topic} (\key{ID}, Name, Description)
	\item[] \textsc{Post} (\key{ID})
	\item[] \textsc{Comment} (\key{\fkey{Post ID}}, \fkey{Parent Post ID}, Body, Points, TimeStamp)
	\item[] \textsc{Question} (\key{\fkey{Post ID}}, Title, Body, Points, TimeStamp)
	\item[] \textsc{Answer} (\key{\fkey{Post ID}}, Accepted, \fkey{Question ID}, Body, Points, TimeStamp)
	\item[] \textsc{Follows} (\key{\fkey{User ID}}, \key{\fkey{Topic ID}})
	\item[] \textsc{Relates to} (\key{\fkey{Topic ID}}, \key{\fkey{Post ID}})
	\item[] \textsc{Vote} (\key{\fkey{User ID}}, \key{\fkey{Post ID}})
\end{itemize}