% for text un key attributes
\newcommand{\key}[1]{\underline{\smash{#1}}}
\newcommand{\fkey}[1]{\textit{#1}}

\subsection{Logical Schema}

We are to convert the ERD in the previous section into a logical schema. First we will convert the entities into tables, and then we will add tables for some of the relations where required.

See the entity tables below. An \key{underlined} value indicates a primary key. An \fkey{italicized} value indicates a foreign key.

\begin{itemize}
	\item[] \textsc{User} (\key{ID}, Name, Email, Password, Points)
	\item[] \textsc{Topic} (\key{ID}, Name, Description)
	\item[] \textsc{Post} (\key{ID})
	\item[] \textsc{Question} (\key{\fkey{Post ID}}, Title, Body, Points, TimeStamp, \fkey{Author ID})
	\item[] \textsc{Answer} (\key{\fkey{Post ID}}, Accepted, \fkey{Question ID}, Body, Points, TimeStamp, \fkey{Author ID})
	\item[] \textsc{Comment} (\key{\fkey{Post ID}}, \fkey{Parent Post ID}, Body, Points, TimeStamp, \fkey{Author ID})
	\item[] \textsc{Follows} (\key{\fkey{User ID}}, \key{\fkey{Topic ID}})
	\item[] \textsc{Relates to} (\key{\fkey{Post ID}}, \key{\fkey{Topic ID}})
	\item[] \textsc{Vote} (\key{\fkey{User ID}}, \key{\fkey{Post ID}})
\end{itemize}