\subsection{Validating the Data}

\subsubsection{Negative Points for Certain Questions}

Since the data we used for populating the questions table was obtained from Stack Exchange, some of the questions had a negative number for the points field. However, unlike Stack Overflow, we do not allow negative points, since we only provide users with the ability to up-vote posts, not to down-vote them.

We therefore ran the following SQL statement to negate all the negative values.

\begin{verbatim}
	UPDATE questions SET points = -points WHERE points < 0;
\end{verbatim}

\subsubsection{Circular References in Comment Hierarchy}

Since we randomly generated the comments, and each comment has a field \verb`post_id`, we ended up with some circular references where a comment was its own parent, or a comment's descendant was also an ancestor of that same comment.

In order to solve this, we exported the data as a CSV which we then modified using spreadsheet software and formulae so that each comment referenced a post with an ID smaller its own. We then truncated the table to remove the random data, and used the text importer to import the processed CSV in the same way which we imported the questions. This CSV file can be found at \verb`/data/tables/comments.csv`